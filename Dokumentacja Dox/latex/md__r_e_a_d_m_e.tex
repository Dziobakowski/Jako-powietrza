\chapter{Jakość-\/powietrza-\/w-\/\+Polsce}
\hypertarget{md__r_e_a_d_m_e}{}\label{md__r_e_a_d_m_e}\index{Jakość-\/powietrza-\/w-\/Polsce@{Jakość-\/powietrza-\/w-\/Polsce}}
\label{md__r_e_a_d_m_e_autotoc_md0}%
\Hypertarget{md__r_e_a_d_m_e_autotoc_md0}%


Projekt\+: Aplikacja do analizy danych o jakości powietrza

Opis\+: Aplikacja służy do analizy danych o jakości powietrza, pobieranych z API Głównego Inspektoratu Ochrony Środowiska (GIOŚ). Umożliwia filtrowanie stacji pomiarowych po nazwie miasta, wyświetlanie danych pomiarowych, analizowanie wyników (min, max, średnia, trend) oraz generowanie wykresów w zadanych okresach czasu.

Wymagania do edycji\+:
\begin{DoxyItemize}
\item wx\+Widgets (do tworzenia GUI)
\item C++11 lub wyższy
\item Biblioteka c\+URL (do pobierania danych z API)
\item Kompilator obsługujący C++11 lub nowszy (np. GCC, MSVC)
\end{DoxyItemize}

Wykorzystanie\+:
\begin{DoxyEnumerate}
\item Pobierz projekt z repozytorium Git\+Hub\+: \href{https://github.com/Dziobakowski/Jako-powietrza}{\texttt{ https\+://github.\+com/\+Dziobakowski/\+Jako-\/powietrza}}
\item Zainstaluj wymagane biblioteki\+:
\begin{DoxyItemize}
\item wx\+Widgets (instrukcja na \href{https://www.wxwidgets.org}{\texttt{ https\+://www.\+wxwidgets.\+org}})
\item c\+URL (instrukcja na \href{https://curl.se/libcurl/}{\texttt{ https\+://curl.\+se/libcurl/}})
\end{DoxyItemize}
\item Skompiluj projekt, używając wybranego kompilatora\+:
\begin{DoxyItemize}
\item Dla GCC\+: g++ -\/o air\+\_\+quality\+\_\+analysis main.\+cpp -\/lwxgtk3.\+0 -\/lcurl
\end{DoxyItemize}
\item Uruchom aplikację\+:
\begin{DoxyItemize}
\item Dla systemów Linux\+: ./air\+\_\+quality\+\_\+analysis
\item Dla systemów Windows\+: air\+\_\+quality\+\_\+analysis.\+exe
\end{DoxyItemize}
\end{DoxyEnumerate}

Funkcje\+:
\begin{DoxyItemize}
\item Pobieranie danych o jakości powietrza z API GIOŚ.
\item Filtrowanie stacji pomiarowych po nazwie miasta.
\item Wyświetlanie danych pomiarowych w postaci tekstowej oraz na wykresie.
\item Analiza danych (minimum, maksimum, średnia, trend).
\item Możliwość filtrowania danych według zakresu czasowego.
\end{DoxyItemize}

W razie problemów lub pytań, skontaktuj się z autorem projektu (proszę, nie \texorpdfstring{$\sim$}{\string~}\+Autor).

Autor\+: Tymon Wolniewicz

Licencja\+: MIT License 